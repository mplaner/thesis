%
%chapter3.tex
%

\chapter{THE DETECTOR AND EXPERIMENTAL APPARATUS}
%Briefly describe CERN (location, policy, history).  
The European Organization for Nuclear Research (CERN) has housed
colliders and high energy physics experiments since its founding in
1954 near Geneva Switzerland.  The organization is currently home to
22 member states, 7 associate members (or working towards the status),
and 5 observer states (including the United States of America).  Many
more countries have scientific contracts or co-operation agreements
with CERN.  The international collaboration at CERN
has contributed many discoveries in the field of particle phyiscs
since its creation.  Currently, the largest experiments at CERN use
the Large Hadron Collider (LHC).  There are 2 general purpose
detectors, A Toroidal LHC ApparatuS (ATLAS) and the Compact Muon
Solenoid (CMS).  Both are located on opposing sides of the LHC ring.
Additional experiments: Large Hadron Collider beauty, A Large Ion
Collider Experiment, Total Elastic and diffractive cross section
Measurement, the Large Hadron Collider forward, and Monopole and
Exotics Detector At the LHC perform various specific measurement
during the LHC's operation.  The following analysis focusses only on
the CMS detector near Cessy, France.

%Briefly list LHC experiments.  ATLAS, CMS, LHCb, ALICE. TOTEM, LHCf, MoEDAL.  
\section{The Large Hadron Collider}
\label{sec:LHC}
The Large Hadron Collider (LHC) is a 26.7 km circumference
proton-proton collider operated by CERN on the border of France and
Switzerland.  The previous CERN accelerator, LEP, tunnel was upgraded
to house the LHC.  The LHC was designed to operate at a luminosity of
$10^{34} cm^{-2}s^{-1}$ with an energy of 7 TeV per proton, giving a
center of mass energy of 14 TeV. The LHC can hold a total of 2808
bunches with a spacing of 25 ns.  During the course of run1, a total
of $6.1fb^{-1}$ data were taken with a center of mass energy of 7TeV and
$23.3 fb^{-1}$ at 8TeV.  The milestone measurement from run1 was the
discovery of the Higgs boson, made concurrently by CMS and ATLAS on
July 4th 2012.  After a hiatus to upgrade the LHC and detectors, the
LHC resumed operation with center of mass collisons of 14 TeV in 2015
and collected a total of XX $fb^{-1}$ in 2015 and YY $fb^{-1}$ in 2016.
%Length, location, history.  Previous LHC run period (discovery of Higgs boson).  Specs of design LHC 1034 cm^-2s^-1.  10^27 (HI).  2808 bunches with 25ns.  7TeV at 44.2pb + 6.1fb.  8TeV 23.3fb.  Run2 conditions: 2015, 2016 datataking.  

In order to produce and collide the high energy protons required at
the LHC, a complex of accelerators is required. The protons originate
in a tank of hydrogen gas that is ionized by the Duo-plasmatron
source.  A radio-frequency cavity accelerates the protons to 750 keV.
The proton beam is then sent to the Linac 2, which accelerates it to
50 MeV. From there, the beam is injected into the Proton Synchrotron
Booster, bring the beam energy up to 1.4 GeV.  The Proton Synchroton
then accelerates the protons up to 25GeV.  The beam is then sent to
the Super Proton Synchrotron where they are brought up to 450 GeV.
From here, the protons are transferred into the two beampipes of the
LHC.  It takes 4 minutes and 20 seconds to inject the protons into the
LHC ring.  This must be done once into the clockwise beampipe and once
into the counterclockwise beampipe.  Then a total of 20 minutes is
required to accelerate the protons up to the current maximum energy of
6.5 TeV.

Typically, the beams circulate and collide inside the detectors for
several hours before dumping the beams and restarting the injection
process.

A slightly different chain of accelerators is required to accelerate
the lead ions and is not discussed here.
%follow the proton through the accelerators to LHC.

A magnetic field of 8.33 T is required to maintain a 7 TeV proton beam
in the LHC.  The magnets are constructed from NbTi superconducting
wire and cooled by liquid helium at 1.9 K.  Dipole magnets are used to
bend the path of the protons to follow the ring.  There are 1232
superconducting dipole magnets in the LHC.  Quadrupole magnets are
used to focus the proton bunches, keeping a tight beam.  They
alternate between squeezing horizontally and vertically.  Finally,
there are XX magnets which focus the beams much tighter at the
collision points, inside the LHC experiments.
%magnet specs  
\section{The Compact Muon Solenoid}
\label{sec:CMS}
The Compact Muon Solenoid is .

A more detailed description of the CMS detector, together with a
definition of the coordinate system used and the relevant kinematic
variables, can be found in~\cite{detectorCMS}.
%CMS specs, list each subdetector + magnet
\subsection{Coordinate system}
\label
CMS uses a right-handed coordinate system, with the origin at the
nominal interaction point, the $x$-axis pointing to the centre of the
LHC, the $y$-axis pointing up (perpendicular to the LHC plane), and
the $z$-axis along the anticlockwise-beam direction. The polar angle,
$\theta$, is measured from the positive $z$-axis and the azimuthal
angle, $\phi$, is measured in the $x$-$y$ plane.
\subsection{The Magnet}

\subsection{The Tracking System}
\subsubsection{The Pixel Detector}
\subsubsection{The Strip Tracker}
\subsection{The Electromagnetic Calorimeter}
The electromagnetic calorimeter consists of 75\,848 lead tungstate
crystals, which provide coverage in pseudorapidity $|\eta| < 1.48
$ in a barrel region (EB) and $1.48 < |\eta| < 3.0$ in two endcap
regions (EE). Preshower detectors consisting of two planes of silicon
sensors interleaved with a total of $3 X_0$ of lead are located in
front of each EE detector.

In the barrel section of the ECAL, an energy resolution of about 1\%
is achieved for unconverted or late-converting photons in the tens of
GeV energy range. The remaining barrel photons have a resolution of
about 1.3\% up to a pseudorapidity of $|\eta| = 1$, rising to
about 2.5\% at $|\eta| = 1.4$. In the endcaps, the resolution of
unconverted or late-converting photons is about 2.5\%, while the
remaining endcap photons have a resolution between 3 and
%4\%~\cite{performancePhotons}.


\subsection{The Hadronic Calorimeter}
\subsection{The Muon System}
\subsubsection{Drift Tubes}
\subsubsection{Cathode Strip Chambers}
\subsubsection{Resistive Plate Chambers}
\subsection{The Trigger System}
Events of interest are selected using a two-tiered trigger
%system~\cite{detectorTrigger}. The first level (L1), composed of
custom hardware processors, uses information from the calorimeters and
muon detectors to select events at a rate of around 100~kHz
within a time interval of less than 4~$\mu$s. The second level, known as
the high-level trigger (HLT), consists of a farm of processors running
a version of the full event reconstruction software optimized for fast
processing, and reduces the event rate to around 1~kHz before
data storage.
\subsubsection{The Level 1 Trigger}
\subsubsection{The High Level Trigger}
\subsection{CMS simulation}
