%%
%% This is file `mplaner_thesis.tex',
%% generated with the docstrip utility.
%%
%% The original source files were:
%%
%% nddiss2e.dtx  (with options: `template')
%% 
%% This is a generated file.
%% 
%%  Copyright (C) 2004-2005 Sameer Vijay
%% 
%%  This file may be distributed and/or modified under the
%%  conditions of the LaTeX Project Public License, either
%%  version 1.2 of this license or (at your option) any later
%%  version. The latest version of this license is in
%%     http://www.latex-project.org/lppl.txt
%% 
%% 
%% ==============================================================
%% 
%% Notre Dame's Dissertation document class by Sameer Vijay
%% that adheres to the University of Notre Dame guidelines
%% published in Spring 2004. Updated by Megan Patnott to adhere
%% to the University of Notre Dame guidelines as of Spring 2013.
%% 
%% Please send any improvements/suggestions to :
%%     Shari Hill, Graduate Reviewer.
%%     shill2@nd.edu
%% 
%% For documentation on how to use nddiss2e class, process the
%% file nddiss2e.dtx through LaTeX.
%% 
%% ==============================================================
%% 
\newcommand{\mh}[1]{\ensuremath{m_H}}
\newcommand{\bestmass}[1]{\ensuremath{125.09~ GeV}}
\newcommand{\intlumi}[1]{\ensuremath{xx fb^{-1}}}
\newcommand{\expsig}[1]{\ensuremath{xx \sigma}}
\ProvidesFile{mplaner_thesis.tex}
    [2013/04/16 v3.2013^^J%
     Template file for NDdiss2e class by Sameer Vijay and updated by Megan Patnott^^J]
\documentclass[draft,twoadvisors]{nddiss2e}
                     % One of the options draft, review, final must be chosen.
                     % One of the options textrefs or numrefs should be chosen
                     % to specify if you want numerical or ``author-date''
                     % style citations.
                     % Other available options are:
                     % 10pt/11pt/12pt (available with draft only)
                     % twoadvisors
                     % noinfo (should be used when you compile the final time
                     %         for formal submission)
                     % sort (sorts multiple citations in the order that they're
                     %       listed in the bibliography)
                     % compress (compresses numerical citations, e.g. [1,2,3]
                     %           becomes [1-3]; has no effect when used with
                     %           the textrefs option)
                     % sort&compress (sorts and compresses numerical citations;
                     %           is identical to sort when used with textrefs)
\usepackage{multirow}
\usepackage{subfigure}
\begin{document}

\frontmatter         % All the items before Chapter 1 go in ``frontmatter''

\title{ MEASUREMENT OF THE HIGGS BOSON DECAY INTO TWO PHOTONS}            % TITLE OF WORK. It must be in all caps, and ensuring this is your
 % responsiblity.
\author{ Michael Planer}           % Author's name
\work{ Dissertation }             % ``Dissertation'' or ``Thesis''
\degaward{ Doctor of Philosophy}         % Degree you're aiming for. Should be one of the following options:
 % ``Doctor of Philosophy'' (do NOT include ``in Subject'')
 % ``Master of Science // in // Subject''
\advisor{ Colin Jessop}          % Advisor's name
\secondadvisor{ Nancy Marinelli} % Second advisor, if used option ``twoadvisors''
\department{ Physics}       % Name of the department

\maketitle           % The title page is created now

 % You must use either the \makecopyright option or the \makepublicdomain option.
 % \copyrightholder{ } % If you're not the copyright holder
 % \copyrightyear{ }   % If the copyright is not for the current year
 % \makecopyright      % If not making your work public domain
                       % uncomment out \makecopyright
\makepublicdomain   % Uncomment this to make your work public domain

 % Including an abstract is optional for a master's thesis, and required for a
 % doctoral dissertation.
\begin{abstract}
The results of the measurement of the Higgs boson decaying into two photons with the 2015 and 2016 dataset is described.  The analysis is performed using the dataset recorded by the CMS experiment at the LHC from pp collisions at center-of-mass energies of 13 TeV corresponding to an integrated luminosity of \intlumi . Event classification was performed to maximize signal efficiency and to study gluon fusion, vector boson fusion, vector boson associated production, and  top fusion Higgs Boson production modes.  Expected sensitivity to the Higgs boson, based on the data-driven estimation of the background is presented.  The analysis shows a median expected significance of \expsig~ for the SM Higgs boson at \bestmass. The observed significance at \mh = \bestmass~ is presented.  A measurement of the mass using the 2016 dataset and the fiducial cross section of the Higgs boson decaying to two photons is also presented.
\end{abstract}
 %                         % Either place the text between begin/end, or
 % \include{abstract}  % put it in a file to be included

 % Including a dedication is optional.
\renewcommand{\dedicationname}{\mbox{}} % Replace \mbox{} if you want
                                           % something else. It must be in
                                           % all caps, and doing so is your
                                           % responsibility.
\begin{dedication}
	To my parents, Norman and Sharon.
\end{dedication}
 %                       % Use one of the two choices to add dedication text
 % \include{dedication}

\tableofcontents
\listoffigures
\listoftables

 % Including a list of symbols is optional.
% \renewcommand{\symbolsname}{newsymname} % Replace ``newsymname'' with
                                            % the name you want, and uncomment
                                            % The name must be in all caps,
                                            % and ensuring this is your
                                            % responsibility
 %\begin{symbols}

% \end{symbols}
 %                       % Use one of the two choices to add symbols text
 % \include{symbols}

 % Including a preface is optional.
 %% \renewcommand{\prefacename}{ } % If you want another Preface name, add
                                   % something else, and uncomment.
                                   % The name must be in all caps, and
                                   % ensuring this is your responsibility.
 % \begin{preface}
 % \end{preface}
 %                       % Use one of the two choices to add preface text
 % \include{preface}

 % Including an acknowledgements section may or may not be optional. It's hard to
 % tell from the information available in Spring 2013.
 %% \renewcommand{\acknowledgename}{ } % If you want another Acknowledgement name
                                       % add something else, and uncomment
                                       % The name must be in all caps, and
                                       % ensuring this is your responsiblity.
\begin{acknowledge}
  I would like to acknowledge my two advisors, Prof. Colin Jessop and
  Research Associate Prof. Nancy Marinelli.  Federico Ferri, Seth
  Zenz, Martina Malberti, Shervin Nourbakhsh (Hgg conveners).  The
  entire Hgg team.  My many office mates in 512: Matthias Wolf, Dr
  Geoff Smith, Fanbo Meng, Dr Yutaro Iiyama who provided both
  technical support and an atmosphere of fun.  My lunchtime group,
  which changed considerably over my time at CERN: Prof. Ted Kolberg,
  Prof. Rachel Yohay, Dr David Morse, Dr Bingxuan Liu, Dr Paul Lujan,
  Juliana Froggatt, Dr Kelly Beernaert, Dr Jess Brinson, and Justin
  Pilot.  My friends at Notre Dame were important to me as well, but
  of particular import is Christopher Wotta.  A more trustworthy
  friend is hard to come by and knowing that I always had a place to
  sleep in South Bend was of no small comfort.  During my time at
  CERN, I was blessed with profound friendships which helped me
  progress both in physics and life.  Dr Sarah Boutle was for a time,
  my only officemate and helped make the first year at CERN a
  delightful experience.  Dr Vera Chetvertkova was a wonderful
  companion on the bus and through many adventures in the Franco-Swiss
  region.  Dr Juliette Alimena's frienship and patience meant a lot to
  me during the final year that I spent at CERN.  Dr Charlie Mueller
  was a brilliant friend throughout graduate school both at Notre Dame
  and at CERN.  And finally, I thank my parents, Norm and Sharon, for
  all their support over the years.  They provided me with every
  opportunity to succeed, while giving me the space to create and
  solve many of my own challenges.  Through their love and guidance
  I've grown from a small child who kept falling out of trees while
  wondering how the world works to an adult who quantifies the
  workings of the world (and who only falls out of trees when he
  intends to).

  \end{acknowledge}
 %                       % Use one of the two choices to add acknowledge text
 % \include{acknowledgement}

\mainmatter

%
% Chapter 1
%

\chapter{INTRODUCTION}
Higgs boson introduction.

% % uncomment the following lines,
% if using chapter-wise bibliography
%
% \bibliographystyle{ndnatbib}
% \bibliography{example}

%
%   Chapter 2
%

\chapter{THEORY}
This chapter will be a basic overview of the standard model of particle physics.
\section{The Standard Model Higgs boson}
\subsection{Overview}
\subsection{Spontaneous electroweak symmetry breaking and the Higgs boson}
\section{Higgs production and decay modes}
\subsection{Higgs production mechanisms}
\subsection{Higgs branching ratios}
\section{Higgs boson properties study in the two photon final state}

%
%chapter3.tex
%

\chapter{THE DETECTOR AND EXPERIMENTAL APPARATUS}
%Briefly describe CERN (location, policy, history).  
The European Organization for Nuclear Research (CERN) has housed
colliders and high energy physics experiments since its founding in
1954 near Geneva Switzerland.  The organization is currently home to
22 member states, 7 associate members (or working towards the status),
and 5 observer states (including the United States of America).  Many
more countries have scientific contracts or co-operation agreements
with CERN.  The international collaboration at CERN
has contributed many discoveries in the field of particle phyiscs
since its creation.  Currently, the largest experiments at CERN use
the Large Hadron Collider (LHC).  There are 2 general purpose
detectors, A Toroidal LHC ApparatuS (ATLAS) and the Compact Muon
Solenoid (CMS).  Both are located on opposing sides of the LHC ring.
Additional experiments: Large Hadron Collider beauty, A Large Ion
Collider Experiment, Total Elastic and diffractive cross section
Measurement, the Large Hadron Collider forward, and Monopole and
Exotics Detector At the LHC perform various specific measurement
during the LHC's operation.  The following analysis focusses only on
the CMS detector near Cessy, France.

%Briefly list LHC experiments.  ATLAS, CMS, LHCb, ALICE. TOTEM, LHCf, MoEDAL.  
\section{The Large Hadron Collider}
\label{sec:LHC}
The Large Hadron Collider (LHC) is a 26.7 km circumference
proton-proton collider operated by CERN on the border of France and
Switzerland.  The previous CERN accelerator, LEP, tunnel was upgraded
to house the LHC.  The LHC was designed to operate at a luminosity of
$10^{34} cm^{-2}s^{-1}$ with an energy of 7 TeV per proton, giving a
center of mass energy of 14 TeV. The LHC can hold a total of 2808
bunches with a spacing of 25 ns.  During the course of run1, a total
of $6.1fb^{-1}$ data were taken with a center of mass energy of 7TeV and
$23.3 fb^{-1}$ at 8TeV.  The milestone measurement from run1 was the
discovery of the Higgs boson, made concurrently by CMS and ATLAS on
July 4th 2012.  After a hiatus to upgrade the LHC and detectors, the
LHC resumed operation with center of mass collisons of 14 TeV in 2015
and collected a total of XX $fb^{-1}$ in 2015 and YY $fb^{-1}$ in 2016.
%Length, location, history.  Previous LHC run period (discovery of Higgs boson).  Specs of design LHC 1034 cm^-2s^-1.  10^27 (HI).  2808 bunches with 25ns.  7TeV at 44.2pb + 6.1fb.  8TeV 23.3fb.  Run2 conditions: 2015, 2016 datataking.  

In order to produce and collide the high energy protons required at
the LHC, a complex of accelerators is required. The protons originate
in a tank of hydrogen gas that is ionized by the Duo-plasmatron
source.  A radio-frequency cavity accelerates the protons to 750 keV.
The proton beam is then sent to the Linac 2, which accelerates it to
50 MeV. From there, the beam is injected into the Proton Synchrotron
Booster, bring the beam energy up to 1.4 GeV.  The Proton Synchroton
then accelerates the protons up to 25GeV.  The beam is then sent to
the Super Proton Synchrotron where they are brought up to 450 GeV.
From here, the protons are transferred into the two beampipes of the
LHC.  It takes 4 minutes and 20 seconds to inject the protons into the
LHC ring.  This must be done once into the clockwise beampipe and once
into the counterclockwise beampipe.  Then a total of 20 minutes is
required to accelerate the protons up to the current maximum energy of
6.5 TeV.

Typically, the beams circulate and collide inside the detectors for
several hours before dumping the beams and restarting the injection
process.

A slightly different chain of accelerators is required to accelerate
the lead ions and is not discussed here.
%follow the proton through the accelerators to LHC.

A magnetic field of 8.33 T is required to maintain a 7 TeV proton beam
in the LHC.  The magnets are constructed from NbTi superconducting
wire and cooled by liquid helium at 1.9 K.  Dipole magnets are used to
bend the path of the protons to follow the ring.  There are 1232
superconducting dipole magnets in the LHC.  Quadrupole magnets are
used to focus the proton bunches, keeping a tight beam.  They
alternate between squeezing horizontally and vertically.  Finally,
there are XX magnets which focus the beams much tighter at the
collision points, inside the LHC experiments.
%magnet specs  
\section{The Compact Muon Solenoid}
\label{sec:CMS}
The Compact Muon Solenoid is .

A more detailed description of the CMS detector, together with a
definition of the coordinate system used and the relevant kinematic
variables, can be found in~\cite{detectorCMS}.
%CMS specs, list each subdetector + magnet
\subsection{Coordinate system}
\label
CMS uses a right-handed coordinate system, with the origin at the
nominal interaction point, the $x$-axis pointing to the centre of the
LHC, the $y$-axis pointing up (perpendicular to the LHC plane), and
the $z$-axis along the anticlockwise-beam direction. The polar angle,
$\theta$, is measured from the positive $z$-axis and the azimuthal
angle, $\phi$, is measured in the $x$-$y$ plane.
\subsection{The Magnet}

\subsection{The Tracking System}
\subsubsection{The Pixel Detector}
\subsubsection{The Strip Tracker}
\subsection{The Electromagnetic Calorimeter}
The electromagnetic calorimeter consists of 75\,848 lead tungstate
crystals, which provide coverage in pseudorapidity $|\eta| < 1.48
$ in a barrel region (EB) and $1.48 < |\eta| < 3.0$ in two endcap
regions (EE). Preshower detectors consisting of two planes of silicon
sensors interleaved with a total of $3 X_0$ of lead are located in
front of each EE detector.

In the barrel section of the ECAL, an energy resolution of about 1\%
is achieved for unconverted or late-converting photons in the tens of
GeV energy range. The remaining barrel photons have a resolution of
about 1.3\% up to a pseudorapidity of $|\eta| = 1$, rising to
about 2.5\% at $|\eta| = 1.4$. In the endcaps, the resolution of
unconverted or late-converting photons is about 2.5\%, while the
remaining endcap photons have a resolution between 3 and
%4\%~\cite{performancePhotons}.


\subsection{The Hadronic Calorimeter}
\subsection{The Muon System}
\subsubsection{Drift Tubes}
\subsubsection{Cathode Strip Chambers}
\subsubsection{Resistive Plate Chambers}
\subsection{The Trigger System}
Events of interest are selected using a two-tiered trigger
%system~\cite{detectorTrigger}. The first level (L1), composed of
custom hardware processors, uses information from the calorimeters and
muon detectors to select events at a rate of around 100~kHz
within a time interval of less than 4~$\mu$s. The second level, known as
the high-level trigger (HLT), consists of a farm of processors running
a version of the full event reconstruction software optimized for fast
processing, and reduces the event rate to around 1~kHz before
data storage.
\subsubsection{The Level 1 Trigger}
\subsubsection{The High Level Trigger}
\subsection{CMS simulation}

%chapter 4

\chapter{ELECTRON AND PHOTON RECONSTRUCTION AND IDENTIFICATION}
\section{Energy measurement in ECAL}
\section{Energy calibration}
\section{Clustering and energy corrections}
\subsection{Parametric electron and photon energy corrections}
\subsection{MultiVariate (MVA) electron and photon energy corrections}
\section{Photon reconstruction}
\subsection{Reconstruction of conversions}
\section{Electron reconstruction}
\section{ECAL noise and simulation}

\chapter{MEASUREMENT OF THE ENERGY SCALE AND ENERGY RESOLUTION}
\section{Intrinsic ECAL energy scale and energy resolution}
\section{Measurement of the \textit{in situ} energy resolution}
\subsection{Contributions to the \textit{in situ} energy resolution}
\subsection{Energy scale and resolution with Zee events}
\subsection{Ztoee event selection}
\subsection{Comparison between data and simulation samples}
\subsection{Fit method}
\subsection{Energy scale correction and experimental resolution estimation}
\subsection{Uncertainties on peak position and experimental resolution}
\section{Smearing method}
\subsection{Mitigation of the likelihood fluctuations}
\subsection{Et dependent energy scale}
\subsection{Minimization algorithm}
\section{Energy scale corrections and additional smearing derivation}
\subsection{Energy scale corrections}
\subsection{Additional smearings}
\subsection{Validation with toy simulation study}
\subsection{Systematics uncertainties and additional smearings}

%chapter 6
\chapter{EVENT SELECTION}
\section{Data samples}
\section{Trigger selection}
\section{Event preselection}
\section{Vertex selection}
\subsection{Conversions}
\subsection{Vertex selection performance}
\subsection{Per-event vertex probability}
\section{Photon Identification}

\chapter{EVENT CATEGORIZATION}
\section{Inclusive categorization}
\subsection{Multivariate diphoton categorization}
\section{Exclusive categories}
\subsection{ttH categories}
\subsection{VH categories}
\subsection{VBF categories}
\section{Signal and background yields}

\chapter{STATISTICAL ANALYSIS}
\section{Background modeling}
\section{Signal modeling}
\section{Systematics}
\section{Final results}

\chapter{CONCLUSIONS}


 % Place the text body here.
 % \include{chapter-one}
 % Begin each chapter with \chapter{TITLE}. Chapter titles must be in all caps
 % and ensuring that they are is your responsibility.

\appendix

 % If you have appendices, add them here.
 % Begin each one with \chapter{TITLE} as before- the \appendix command takes
 % care of renaming chapter headings and creates a new page in the Table of
 % Contents for them.
 % \include{appendix-one}

\backmatter              % Place for bibliography and index


%\bibliographystyle{nddiss2e}
\bibliographystyle{unsrtnat}
\bibliography{mplaner_thesis}           % input the bib-database file name


\end{document}

%%
\endinput
%%
%% End of file `template.tex'.
